\section{Related Work}

\begin{frame}[plain]
    \sectionpage
\end{frame}
%
% \begin{frame}
%     \frametitle{eBPF-based Monitoring and Observability Solutions}
%     \begin{itemize}
%         \item Burns \cite{Burns2018} introduces the sidecar pattern for observability in Kubernetes
%             \begin{itemize}
%                 \item Resource overhead
%                 \item Manual instrumentation of each service
%             \end{itemize}
%         \item Rice \cite{Rice2022} proposes a paradigm shift toward using a single eBPF-based agent per node
%             \begin{itemize}
%                 \item Tracee \cite{TraceeOnline} forensics tool
%             \end{itemize}
%         \item Nam et al. \cite{Nam2022} extend eBPF applications to inter-container communication security
%             \begin{itemize}
%                 \item Addressing unauthorized access prevention between containers sharing the same host
%                 \item It focuses primarily on communication channels rather than comprehensive behavioral analysis of containerized applications
%             \end{itemize}
%         % \item Current eBPF-based monitoring solutions show substantial performance enhancements compared to traditional methods
%         %     \begin{itemize}
%         %         \item However, most of these solutions function as standalone tools or agents and do not integrate well with Kubernetes
%         %     \end{itemize}
%     \end{itemize}
% \end{frame}

% \begin{frame}
%     \frametitle{System Call Analysis}
%     \begin{itemize}
%         \item Abed et al. \cite{Abed2015} conducted pioneering research applying BoSC techniques to identify behavioral anomalies in Linux containers
%             \begin{itemize}
%                 \item Collect system call traces using the \textit{strace} tool and comparing them against pre-established normal behavior baselines
%                 \item Reactive nature, i.e., anomalies are identified only after attacks have occurred
%             \end{itemize}
%         \item Castanhel et al. \cite{Castanhel2021} investigated system call sequence analysis for detecting security threats in containerized applications
%             \begin{itemize}
%                 \item Machine learning algorithms to evaluate system call sequences against custom datasets containing both normal and anomalous behaviors
%             \end{itemize}
%     \end{itemize}

%     % While analyzing system calls has shown potential for detecting anomalies in containerized applications, current methods have several limitations.
%     % The primary emphasis on detecting incidents after they occur reduces the effectiveness of real-time threat prevention.
%     % Furthermore, existing research lacks evaluation against modern container-specific evasion techniques.
% \end{frame}


% \begin{frame}
%     \frametitle{Network Traffic Analysis and Anomaly Detection}
%     \begin{itemize}
%         \item Liu et al. \cite{Liu2020} developed a comprehensive system utilizing eBPF to analyze network traffic across containerized environments
%             \begin{itemize}
%                 \item Three distinct eBPF program types to collect data from various network protocols: socket filters, kprobe/kretprobe, and tracepoint.
%                 \item Machine learning techniques for anomaly detection
%             \end{itemize}
%             % While this low overhead is promising, the specific testing scenarios and workload characteristics that produced these results require further examination for broader applicability.

%         \item Shiraishi et al. \cite{Shiraishi2020} focus specifically on real-time monitoring within containerized microservice architectures
%             \begin{itemize}
%                 \item Propose a system that utilizes eBPF sensors to gather service-related network metrics
%             \end{itemize}
%             % Their approach constructs comprehensive full-stack topology views and dynamically adjusts sensor positioning in response to container orchestrator events.
%             % However, their work primarily addresses operational observability rather than security-oriented anomaly detection.

%         \item Lee et al. \cite{Lee2022} address the specific challenges of packet tracing in container overlay networks through eBPF-based solutions
%             \begin{itemize}
%                 \item Enhance traditional distributed tracing methods by incorporating infrastructure-layer analysis alongside application-layer latency measurements
%             \end{itemize}
%             % To overcome existing tracing limitations, they employ a modified sidecar proxy to insert trace context at fixed positions within HTTP headers, effectively reducing the search space required for trace context detection using eBPF.
%             % While innovative, this approach requires application-level modifications that may limit its applicability in environments where application changes are not feasible.

%         \item Sharma et al. \cite{Sharma2024} presents an eBPF-based observability solution specifically designed for cloud-native applications in Kubernetes environments
%             \begin{itemize}
%                 \item Solution operates entirely in kernel space without degrading the performance of instrumented events
%             \end{itemize}
%             % The authors report significant performance improvements compared to existing solutions, with cluster CPU usage reduced by factors ranging from 21 to 214, and memory utilization decreased by 30\% to 159\%.
%             % However, these significant performance claims require validation under diverse workload conditions.

%             % Current methods utilizing network-level eBPF monitoring provide various options for monitoring network traffic and detecting anomalies.
%             % However, they frequently lack a real-time capability to identify issues such as unusual traffic spikes.
%             % Additionally, these approaches often depend on modifications at the application level, which may not always be feasible.

%             % Our work seeks to address this gap by introducing a solution that continuously and passively monitors all incoming traffic at the pod level, utilizing a custom statistical algorithm to identify deviations in network activity, and promptly issues alerts for such occurrences. This approach enhances early detection capabilities, allowing for a proactive response to potential security threats or system inefficiencies.
%     \end{itemize}
% \end{frame}

% todo: multi domian foucus

\begin{frame}
    \frametitle{Research Gaps and Opportunities}
    \begin{table}[htbp]
        \centering
        % \caption{State-of-the-art in Container Security Monitoring}
        \label{tab:sota-container-security}
        \resizebox{\textwidth}{!}{%
            \begin{tabular}{|l|l|l|l|}
                \hline
                \textbf{Study} & \textbf{Approach} & \textbf{Focus} & \textbf{Real-time} \\
                \hline
                Burns (2018) & Sidecar pattern & General observability & Yes \\
                \hline
                Rice (2022) & eBPF agent & Host-level tracing & Yes \\
                \hline
                Cassagnes (2020) & eBPF & Performance monitoring & Yes \\
                \hline
                Nam (2022) & eBPF & Inter-container communication & Yes \\
                \hline
                Abed (2015) & strace + BoSC & Anomaly detection & No \\
                \hline
                Castanhel (2021) & ML + syscalls & Privilege escalation & No \\
                \hline
                Liu (2020) & eBPF + ML & Network monitoring & Yes \\
                \hline
                Shiraishi (2020) & eBPF sensors & Network monitoring & Yes \\
                \hline
                Lee (2022) & eBPF + sidecar & Packet tracing & Yes \\
                \hline
                Sharma (2024) & eBPF & Cloud-native observability & Yes \\
                \hline
                \textbf{This work} & \textbf{eBPF} & \textbf{Multi-domain} & \textbf{Yes} \\
                \hline
            \end{tabular}%
        }
    \end{table}
\end{frame}

\begin{frame}
    \frametitle{Research Gaps and Opportunities}
    \textbf{Existing solutions:}
    \begin{itemize}
        \item Monitor single domain (network or syscall)
        \item Reactive: detect anomalies post-incident
        \item Lack Kubernetes-native integration
    \end{itemize}

    \vspace{1em}
    \textbf{This thesis:}
    \begin{itemize}
        \item Multi-domain: shell sessions, system calls, and network
        \item Proactive real-time detection
        \item Native Kubernetes integration (2 case studies)
    \end{itemize}
\end{frame}


