\section{Introduction}

\begin{frame}
    \frametitle{Introduction}
    \begin{itemize}
        \item Containers and container orchestrators (CO) increasingly popular
            % \item  Cloud Native Computing Foundation (CNCF) and Datadog surveys \cite{CNCFSurvey2024,DatadogSurvey2022}:
            %     \begin{itemize}
            %         \item \textbf{52\%}: containers in almost all production systems
            %         \item \textbf{39\%}: containers in some production systems
            %         \item \textbf{6\%}: actively evaluating
            %         \item \textbf{Kubernetes}: used by ~50\% of organizations for managing containers
            %     \end{itemize}
        \item Popularity $\rightarrow$ new security challenges
            % \begin{itemize}
            %     \item ~50\% of organizations faced security breach in 2022 \cite{AquaReport2023}
            % \end{itemize}
        \item Containers share host kernel
            \begin{itemize}
                % \item Lightweight runtime
                \item Less secure than virtual machines (VM)
            \end{itemize}
        \item Additional security mechanisms needed
            \begin{itemize}
                \item Example: runtime monitoring
            \end{itemize}
    \end{itemize}
\end{frame}

%%%%%%%%%%%%%%%%%%%%%%%%%%%%%%%%%%%%%%%%%%%%%

% TODO: removed for space

% \begin{frame}
%   \frametitle{Introduction}
%   \begin{itemize}
%   \item A Linux container is a process with a higher level of isolation.
%   \item Isolation mechanisms:
%     \begin{itemize}
%     \item \textbf{Namespaces} isolate various system resources, such as the network stack and process tree, ensuring that containers cannot interact with each other's processes \cite{Simonsson2021}.
%     \item \textbf{cGroups}: control the allocation of hardware resources such as RAM and CPU to each container.
%     \item \textbf{Seccomp}: restricts the system calls that a container can make, thus limiting its access to the host system.
%     \end{itemize}
%   \end{itemize}
% \end{frame}

%%%%%%%%%%%%%%%%%%%%%%%%%%%%%%%%%%%%%%%%%%%%%

% \begin{frame}
%     \frametitle{Introduction}
%     \begin{itemize}
%         \item Popularity $\rightarrow$ new security challenges
%             \begin{itemize}
%                 \item ~50\% of organizations faced security breach in 2022 \cite{AquaReport2023}
%             \end{itemize}
%         \item Containers share host kernel
%             \begin{itemize}
%                 % \item Lightweight runtime
%                 \item Less secure than virtual machines (VM)
%             \end{itemize}
%         \item Additional security mechanisms needed
%         \begin{itemize}
%             \item Example: runtime monitoring
%         \end{itemize}
%     \end{itemize}
% \end{frame}

%%%%%%%%%%%%%%%%%%%%%%%%%%%%%%%%%%%%%%%%%%%%%

\begin{frame}
    \frametitle{Introduction}
    \begin{itemize}
        % \item To improve the security of containers: runtime monitoring
        % \item Monolithic applications
        %     \begin{itemize}
        %         \item Easier to monitor
        %         \item Single host
        %     \end{itemize}
        % \item Cloud-native applications
        %     \begin{itemize}
        %         \item Distributed across various hosts
        %         \item Modifying the application for instrumentation is hard
        %     \end{itemize}
        \item Most observability approaches: network or app-level metrics
            % \begin{itemize}
            %     \item Sidecars
            %     \item Node-level agents
            %     \item Application-level instrumentation
            % \end{itemize}
        \item Lack of system-level observability
            \begin{itemize}
                \item Kernel-level events
            \end{itemize}
            % \item It is important to develop solutions that collect data from the low-level host operating system
            % \begin{itemize}
            %     \item Provide fine-grained, low-overhead visibility into system behavior
            % \end{itemize}
        \item Extended BSD Packet Filter (eBPF)
            % \begin{itemize}
            % \item Executes bytecode at kernel hook points
            % \item Used for networking, tracing, and security
            % \end{itemize}
        \item Monitoring containers with eBPF is challenging
            \begin{itemize}
                \item Container isolation
            \end{itemize}
        \item \textbf{Goal}: examine eBPF for container security
            % \begin{itemize}
            %     \item Monitor container activity in real time
            %     \item Identify malicious behavior within containers
            %     \item Assess overhead of eBPF-based solutions
            % \end{itemize}
    \end{itemize}
\end{frame}

%%%%%%%%%%%%%%%%%%%%%%%%%%%%%%%%%%%%%%%%%%%%%%%%%%%%%

% \begin{frame}
%     \frametitle{Introduction}
%     \begin{itemize}
%     \end{itemize}
% \end{frame}

%%%%%%%%%%%%%%%%%%%%%%%%%%%%%%%%%%%%%%%%%%%%%%%%%%%%%


