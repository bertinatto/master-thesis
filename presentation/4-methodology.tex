\section{Methodology}

\begin{frame}[plain]
    \sectionpage
\end{frame}

\begin{frame}
    \frametitle{Research Approach}
    \begin{itemize}
        \item Exploratory multi-case study approach
        \item Why multiple case studies?
            \begin{itemize}
                \item eBPF technical diversity
                \item Containers need broad security coverage
                \item Deployment contexts
                % \item Real-world applicability
            \end{itemize}
    \end{itemize}
\end{frame}

% todo: removever o how and why da dissertacao

% \begin{frame}
%     \frametitle{Methodology: Research Approach}
%     \begin{itemize}
%         \item Qualitative, exploratory research design, with multiple case studies as the primary research method
%         \begin{itemize}
%             \item Case study research is particularly suitable for investigating contemporary phenomena within their real-life context\cite{Yin2018}
%             \item Ideal for examining emerging technologies, such as eBPF, in production container environments
%         \end{itemize}
%         \item Iterative approach where each case study builds upon insights from previous cases
%     \end{itemize}
% \end{frame}

% \begin{frame}
%     \frametitle{Methodology: Research Approach}
%     \begin{itemize}
%         \item The multiple case study approach was selected for several strategic reasons
%         \item Technical diversity offered by eBPF
%         \begin{itemize}
%             \item Various program types that can be attached to different kernel hook points
%         \end{itemize}
%         \item Security coverage that eBPF can provide
%         \begin{itemize}
%             \item Container security encompasses multiple attack vectors due to the shared kernel architecture
%         \end{itemize}
%         \item Deployment contexts
%         \begin{itemize}
%             \item Standalone container deployments
%             \item Complex orchestrated systems like Kubernetes
%         \end{itemize}
%         \item Real-world applicability
%         \item The research framework follows a systematic approach for each case study
%         \begin{itemize}
%             \item Identify a specific container security challenge
%             \item An eBPF-based solution is then designed to address the identified problem
%             \item Implementation of the proposed solution
%             \item Functional validation to assess effectiveness
%             \item Performance impact assessment, when it makes sense
%         \end{itemize}
%     \end{itemize}
% \end{frame}

\begin{frame}
    \frametitle{Research Approach}   
    \begin{itemize}
        \item Systematic approach per each case study
            \begin{itemize}
                \item Identify security challenge
                \item Design eBPF-based solution
                \item Implement solution
                \item Validate effectiveness
                \item Assess performance impact when it makes sense
            \end{itemize}
        \item Each case study builds on previous insights
    \end{itemize}
\end{frame}

\begin{frame}
    \frametitle{Research Approach}
    \begin{itemize}
        \item Implementation
            \begin{itemize}
                \item Dual-component architecture
                \item User space components
                    \begin{itemize}
                        % \item Python + BCC unavailable in containers
                        \item Go + Cilium eBPF library\footnote{https://github.com/cilium/ebpf/tree/main}
                        \item Attach eBPF programs to hook points in the kernel
                        \item Handle Kubernetes integration
                    \end{itemize}
                \item Kernel space components
                    \begin{itemize}
                        \item Written in C
                        \item Compiled with LLVM clang
                    \end{itemize}
            \end{itemize}
    \end{itemize}
\end{frame}

% \begin{frame}
%     \frametitle{Methodology: Research Approach}
%     \begin{itemize}
%         \item Deployment
%         \item Case study-dependent
%         \begin{itemize}
%             \item eBPF-based solutions injected into containers
%             \item Single eBPF-based program monitored the containerized applications from the host system
%         \end{itemize}
%         \item In all cases, no changes were required to the monitored applications
%     \end{itemize}
% \end{frame}

% \begin{frame}
%     \frametitle{Methodology: Research Approach}
% % \section{Data Collection}
%     \begin{itemize}
%         \item Data collection
%         \begin{itemize}
%             \item Performed according to the environment and goals of each case study
%             \item Several programs were used
%             \item Programs available in the Linux kernel source tree
%             \item Custom and existing benchmarking tools
%             \item All datasets produced during benchmarking were retained for analysis and are publicly accessible \footnote{https://github.com/bertinatto/master-thesis/tree/master/src}
%         \end{itemize}
%     \end{itemize}
% \end{frame}

% \begin{frame}
%     \frametitle{Methodology: Tools And Technologies}
%     \begin{table}
%         \centering
%         \begin{tabular}{|p{0.32\linewidth}|p{0.6\linewidth}|}
%             % \begin{tabular}{|l|l|}
%             \hline
%             \textbf{Tool}             & \textbf{Purpose}                                        \\ \hline
%             Podman, Containerd, CRI-O & Container creation and management                       \\ \hline
%             Kubernetes, OpenShift     & Container orchestration                                 \\ \hline
%             Vagrant                   & Virtualization of Kubernetes nodes                      \\ \hline
%             ebpf-go library           & Creation of eBPF-based solutions                        \\ \hline
%             Nginx, Redis              & Containerized application workloads                     \\ \hline
%             ab, redis-benchmark       & Performance benchmarking and functional testing         \\ \hline
%             R language                & Data analysis                                           \\ \hline
%         \end{tabular}
%         % \caption{Tools and Technologies Used in This Research.}
%         \label{tab:tools}
%     \end{table}
% \end{frame}

